\chapter{Исследовательская часть}

В данном разделе приведены примеры работы программы и анализ характериситик разработанного программного обеспечения.

\section{Технические характеристики}

\begin{itemize}
	\item Операционная система: Manjaro \cite{manjaro} Linux \cite{linux} x86\_64.
	\item Память: 8 ГБ.
	\item Процессор: Intel® Core™ i7-8550U\cite{intel}.
\end{itemize}

Тестирование проводилось на ноутбуке, включенном в сеть электропитания. Во время тестирования ноутбук был нагружен только встроенными приложениями окружения, окружением, а также непосредственно системой тестирования.

\section{Пример работы и анализ результата}

Пример работы программы приведен на рисунке \ref{img:example}.
\clearpage
\img{200mm}{example}{Пример работы программы}
\clearpage

Из примера работы видно, что первый этап является самым трудоёмким. Время его работы достаточно стабильно, так как в нём рассчитываются хэш-значения для всех подстрок данной строки, и примерно равняется 0.4 секунды в среднем при длине строк 200000 и длине подстроки 200.

Второй этап выполняется быстрее всех и тоже достаточно стабилен, так как в нем идёт расчет хэш-значения для подстроки, которую надо найти в данной строке, а также сравнение полученного хэш-значения для подстроки и всех хэш-значений последовательных подстрок в данной строке. Среднее время работы примерно равно 0.01 секунды.

Третий этап нестабилен, потому что количество сравниваемых строк в нём равно количеству хэш-значений, которые совпали на втором этапе. Среднее время работы данного этапа примерно равняется 0.2 секунды.

\section*{Вывод}

Второй этап конвейера получился наименее трудоёмким в среднем (0.01 секунды), а первый - наиболее долгим в среднем (0.4 секунды). Третий этап выполняется в среднем за 0.2 секунды, однако при большем количестве совпадающих подстрок данной строки с данной подстрокой это время может увеличиваться и быть большем, нежели время первого этапа, то есть оно является нестабильным и зависит от данных. Все данные приведены для входных данных, в которых длина строки равна 200000, а длина искомой подстроки равна 200.
