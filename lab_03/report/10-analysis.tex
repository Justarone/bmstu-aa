\chapter{Аналитическая часть}

\section{Сортировка пузырьком}

Алгоритм состоит из повторяющихся проходов по сортируемому массиву.
За каждый проход элементы последовательно сравниваются попарно и, если порядок в паре неверный, выполняется обмен элементов.
Проходы по массиву повторяются $N-1$ раз, но есть модифицированная версия, где если окажется, что обмены больше не нужны, значит проходы прекращаются.
При каждом проходе алгоритма по внутреннему циклу очередной наибольший элемент массива ставится на свое место в конце массива рядом с предыдущим ``наибольшим элементом'', а наименьший элемент массива перемещается на одну позицию к началу массива (``всплывает'' до нужной позиции, как пузырёк в воде -- отсюда и название алгоритма).

\section{Сортировка вставками}

Сортировка вставками — алгоритм сортировки, котором элементы входной последовательности просматриваются по одному, и каждый новый поступивший элемент размещается в подходящее место среди ранее упорядоченных элементов \cite{Knut}.

В начальный момент отсортированная последовательность пуста.
На каждом шаге алгоритма выбирается один из элементов входных данных и помещается на нужную позицию в уже отсортированной последовательности до тех пор, пока набор входных данных не будет исчерпан.
В любой момент времени в отсортированной последовательности элементы удовлетворяют требованиям к выходным данным алгоритма.

\section{Сортировка выбором}

Шаги алгоритма:
\begin{enumerate}
	\item находим номер минимального значения в текущем массиве;
	\item производим обмен этого значения со значением первой неотсортированной позиции (обмен не нужен, если минимальный элемент уже находится на данной позиции);
	\item теперь сортируем ``хвост'' массива, исключив из рассмотрения уже отсортированные элементы.
\end{enumerate}

Для реализации устойчивости алгоритма необходимо в пункте 2 минимальный элемент непосредственно вставлять в первую неотсортированную позицию, не меняя порядок остальных элементов, что может привести к резкому увеличению числа обменов. 

\section*{Вывод}
В данной работе стоит задача реализации 3 алгоритмов сортировки, а именно: пузырьком, вставками и выбором.
Необходимо оценить теоретическую оценку алгоритмов и проверить ее экспериментально.
