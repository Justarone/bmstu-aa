\chapter*{Введение}
\addcontentsline{toc}{chapter}{Введение}
\textbf{Расстояние Левенштейна} - это минимальное количество операций, необходимых для превращения одной строки в другую, где операции:
\begin{itemize}
    \item вставкa одного символа; 
    \item удалениe одного символа;
    \item замены одного символа на другой.
\end{itemize}

Расстояние Левенштейна находит практическое применение во многих сферах:

\begin{itemize}
	\item алгоритмы нечеткого поиска
	\item сравнения текстовых файлов
	\item сравнения генов, хромосом и белков в биоинформатике
\end{itemize}

Целью данной лабораторной работы является реализация рекурсивных и итеративных алгоритмов Левенштейна и Дамерау-Левенштейна. 

Задачами данной лабораторной являются:
\begin{enumerate}
  	\item изучение алгоритмов Левенштейна и Дамерау-Левенштейна нахождения расстояния между строками;
	\item применение техник динамического программирования для реализации указанных алгоритмов; 
    \item получение практических навыков реализации указанных алгоритмов: двух итеративных алгоритмов (Левенштейна и Дамерау-Левенштейна) и алгоритма Левенштейна в 2 рекурсивных версиях (с мемоизацией и без нее); 
	\item сравнительный анализ итеративной и рекурсивной реализаций выбранного алгоритма определения расстояния между строками по затрачиваемым ресурсам (времени и памяти); 
	\item экспериментальное подтверждение различий во временнóй эффективности различных реализаций исследуемых алгоритмов определения расстояния между строками при помощи разработанного программного обеспечения на материале замеров процессорного времени выполнения реализации на варьирующихся длинах строк; 
	\item описание и обоснование полученных результатов в отчете о выполненной лабораторной работе. 
\end{enumerate}
