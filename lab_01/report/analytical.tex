\chapter{Аналитическая часть}
Задача по нахождению расстояния Левенштейна заключается в поиске минимального количества операций вставки/удаления/замены для превращения одной строки в другую.

При нахождении расстояния Дамерау — Левенштейна добавляется операция транспозиции (перестановки соседних символов).

\textbf{Действия обозначаются следующим образом:}
\begin{enumerate}
    \item D (англ. delete) --- удалить,
    \item I (англ. insert) --- вставить,
    \item R (replace) --- заменить,
    \item M (match) --- совпадение.
\end{enumerate}

Пусть $S_{1}$ и $S_{2}$ — две строки (длиной M и N соответственно) над некоторым алфавитом, тогда расстояние Левенштейна можно подсчитать по следующей рекуррентной формуле:

\begin{displaymath}
    D(i,j) = \left\{ \begin{array}{ll}
        0, & \textrm{$i = 0, j = 0$}\\
        i, & \textrm{$j = 0, i > 0$}\\
        j, & \textrm{$i = 0, j > 0$}\\
        min(\\
        D(i,j-1)+1,\\
        D(i-1, j) +1, &\textrm{$j>0, i>0$}\\
        D(i-1, j-1) + m(S_{1}[i], S_{2}[j])\\
        ),
    \end{array} \right.
\end{displaymath}

где $m(a,b)$ равна нулю, если $a=b$ и единице в противном случае; $min\{\,a,b,c\}$ возвращает наименьший из аргументов.

Расстояние Дамерау-Левенштейна вычисляется по следующей рекуррентной формуле:

\[ D(i, j) =  \left\{
    \begin{aligned}
        &0, && i = 0, j = 0\\
        &i, && i > 0, j = 0\\
        &j, && i = 0, j > 0\\
        &min \left\{
            \begin{aligned}
                &D(i, j - 1) + 1,\\
                &D(i - 1, j) + 1,\\
                &D(i - 1, j - 1) + m(S_{1}[i], S_{2}[i]), \\
                &D(i - 2, j - 2) + m(S_{1}[i], S_{2}[i]),\\
            \end{aligned} \right.
            &&
            \begin{aligned}
                &, \text{ если } i, j > 1 \\
                & \text{ и } S_{1}[i] = S_{2}[j - 1] \\
                & \text{ и } S_{1}[i - 1] =  S_{2}[j] \\
            \end{aligned} \\
            &min \left\{
                \begin{aligned}
                    &D(i, j - 1) + 1,\\
                    &D(i - 1, j) + 1, \\
                    &D(i - 1, j - 1) + m(S_{1}[i], S_{2}[i]),\\
                \end{aligned} \right.  &&, \text{иначе}
    \end{aligned} \right.
    \]

\section{Вывод}
В данном разделе были рассмотрены алгоритмы нахождения расстояния Левенштейна и Дамерау-Левенштейна, который является модификаций первого, учитывающего возможность перестановки соседних символов.
