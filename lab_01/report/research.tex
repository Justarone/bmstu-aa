\chapter{исследовательская часть}

\section{сравнительный анализ на основе замеров времени работы алгоритмов}

был проведен замер времени работы каждого из алгоритмов.


\begin{table} [h!]
\caption{время работы алгоритмов (в тиках)}
	\begin{tabular}{|c c c c c|} 
 	\hline
	len & lev(r) & lev(t) & damlev(r) & damlev(t) \\ [0.8ex] 
 	\hline\hline
 	1 & 5928 & 3724258 & 7456 & 4367560\\
 	\hline
 	2 & 16865 & 7224736 & 21854 & 8286833\\
 	\hline
	3 & 62333 & 12123365 & 105445 & 12852145\\
	\hline
	4 & 372661 & 16940041 & 407763 & 18585284\\
	\hline
	5 & 1909255 & 23402008 & 1966658 & 24103230\\
	\hline
	6 & 9065189 & 32328258 & 11002094 & 27935583\\
	\hline
	7 & 453325069 & 30166031 & 51219656 & 30567571\\
	\hline
	\end{tabular}
\end{table}


\begin{tikzpicture}

\begin{axis}[
    	axis lines = left,
    	xlabel={len (symbols)},
    	ylabel={time (ticks)},
    	xmin=5, xmax=200,
    	ymin=1, ymax=600000,
	legend pos=north west,
	ymajorgrids=true
]
\addplot[color=red] table[x index=0, y index=1] {iter.dat}; 
\addplot[color=orange] table[x index=0, y index=1] {iterdl.dat};
\addplot[color=blue, mark=square] table[x index=0, y index=1] {rec.dat};
\addplot[color=green, mark=square] table[x index=0, y index=1] {recwithmem.dat};

\addlegendentry{iterative}
\addlegendentry{iterativedl}
\addlegendentry{recursive}
\addlegendentry{recursivewithmem}
\end{axis}
\end{tikzpicture}


\par
наиболее эффективными по времени при маленькой длине слова являются рекурсивные реализации алгоритмов, но как только увеличивается длина слова, их эффективность резко снижается, что обусловлено большим количеством повторных рассчетов. время работы алгоритма, использующего матрицу, намного меньше благодаря тому, что в нем требуется только (m + 1)*(n + 1) операций заполнения ячейки матрицы. также установлено, что алгоритм дамераулевенштейна работает немного дольше алгоритма левенштейна, т.к. в нем добавлены дополнительные проверки, однако алгоритмы сравнимы по временной эффективности.

\section{тестовые данные}

\par
проведем тестирование программы. в столбцах "ожидаемый результат" и "полученный результат" 4 числа соответсвуют рекурсивному алгоритму нахождения расстояния левенштейна, матричному алгоритму нахождению расстоянию левенштейна, рекурсивному алгоритму расстояния дамерау-левенштейна, матричному алгоритму нахождения расстояние дамерау-левенштейна.

\begin{table} [h!]
\caption{таблица тестовых данных}
	\begin{tabular}{|c c c c c c|} 
 	\hline
    № & первое слово & второе слово & описание & ожидание & результат \\ [0.7ex] 
 	\hline\hline
    1 & добро & нутро & только замены & 3 3 3 3 & 3 3 3 3\\
 	\hline
    2 & бро & добро & только удаления & 2 2 2 2 & 2 2 2 2\\
 	\hline
    3 & доброта & добро & только вставки & 2 2 2 2 & 2 2 2 2\\
	\hline
    4 & доброта & одброат & только смены мест & 4 4 4 2 & 4 4 4 2\\
	\hline
    5 & дброта & добрта & вставки и удаления & 2 2 2 2 & 2 2 2 2\\
	\hline
    6 & изменить & подмена & все вместе & 6 6 6 6 & 6 6 6 6\\
	\hline
	\end{tabular}
\end{table}



