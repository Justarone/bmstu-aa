\chapter{Исследовательская часть}

\section{Сравнительный анализ на основе замеров времени работы алгоритмов}

Был проведен замер времени работы каждого из алгоритмов.


\begin{table} [h!]
\caption{время работы алгоритмов (в наносекундах)}
	\begin{tabular}{|c c c c c|} 
 	\hline
	len & recursive & recWithMem & iter & iterDl \\ [0.8ex] 
 	\hline\hline
 	10 & 32766430 & 1313 & 634 & 681\\
 	\hline
    20 & $> 10^{10}$ & 5157 & 2367 & 2582\\
 	\hline
    30 & $> 10^{10}$ & 11342 & 4813 & 5207\\
	\hline
	50 & $> 10^{10}$ & 30066 & 12518 & 13533\\
	\hline
    100 & $> 10^{10}$ & 116134 & 48111 & 54078\\
	\hline
    200 & $> 10^{10}$ & 529335 & 249057 & 527797\\
	\hline
	\end{tabular}
\end{table}


\begin{tikzpicture}

\begin{axis}[
    	axis lines = left,
    	xlabel={len (symbols)},
    	ylabel={time (ticks)},
    	xmin=5, xmax=200,
    	ymin=1, ymax=600000,
	legend pos=north west,
	ymajorgrids=true
]
\addplot[color=red] table[x index=0, y index=1] {iter.dat}; 
\addplot[color=orange] table[x index=0, y index=1] {iterdl.dat};
\addplot[color=blue, mark=square] table[x index=0, y index=1] {rec.dat};
\addplot[color=green, mark=square] table[x index=0, y index=1] {recwithmem.dat};

\addlegendentry{iterative}
\addlegendentry{iterativedl}
\addlegendentry{recursive}
\addlegendentry{recursivewithmem}
\end{axis}
\end{tikzpicture}


\par
Из экспериментов видно, что без применения техник динамеческого программирования даже на сравнительно небольших длинах строк (20) алгоритм выдает сильно худшие результаты в сравнении с алгоритмами, запоминающими результаты своих вычислений.

\section{Тестовые данные}

\par
Проведем тестирование программы. в столбцах "Ожидание" и "Результат"~4 числа соответствуют рекурсивному алгоритму нахождения расстояния Левенштейна, рекурсивному алгоритму с мемоизацией нахождения расстояния левенштейна, итеративному алгоритму нахождения расстояния Левенштейна и итеративному алгоритму нахождения расстояния Дамерау-Левенштейна.

\begin{table} [h!]
\caption{Таблица тестовых данных}
	\begin{tabular}{|c c c c c c|} 
 	\hline
    № & Первое слово & Второе слово & Описание & Ожидание & Результат \\ [0.8ex] 
 	\hline\hline
    1 & добро & нутро & только замены & 3 3 3 3 & 3 3 3 3\\
 	\hline
    2 & бро & добро & только удаления & 2 2 2 2 & 2 2 2 2\\
 	\hline
    3 & доброта & добро & только вставки & 2 2 2 2 & 2 2 2 2\\
	\hline
    4 & доброта & одброат & только смены мест & 4 4 4 2 & 4 4 4 2\\
	\hline
    5 & дброта & добрта & вставки и удаления & 2 2 2 2 & 2 2 2 2\\
	\hline
    6 & изменить & подмена & все вместе & 6 6 6 6 & 6 6 6 6\\
	\hline
	\end{tabular}
\end{table}



