\chapter*{Заключение}
\addcontentsline{toc}{chapter}{Заключение}
В рамках данной лабораторной работы были изучены и реализованы рекурсивные и итеративные версии алгоритмов Левенштейна и Дамерау-Левенштейна.

Также были при переходе от рекурсивного алгоритма к рекурсивного с мемоизацией были применены техники динамического программирования, что позволило заметно улучшить скорость работы алгоритма.

Было сравнительно показано и экспериментально подтверждено различие во временной эффективности рекурсивной и нерекурсивной реализаций выбранного алгоритма определения расстояния между строками при помощи разработаного программного обеспечения на материале замеров процессорного времени выполнения реализации на варьирующихся длинах строк. 

В результате исследований становится очевидно, что алгоритмы, не использующие техники динамического программирования делают очень много лишней работы, в следствие чего работают значительно медленнее, чем алгоритмы, использующие соответствующие техники. В реальной жизни все задачи, имеющие рекурсивное соотношение, так или иначе используют техники динамического программирования.
