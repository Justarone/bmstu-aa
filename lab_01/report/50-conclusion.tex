\chapter*{Заключение}
\addcontentsline{toc}{chapter}{Заключение}

В ходе выполнения работы были выполнены все поставленные задачи и изучены методы динамического программирования на основе алгоритмов вычисления расстояния Левенштейна.

Экспериментально были установлены различия в производительности различных алгоритмов нахождения расстояния Левенштейна. Рекурсивный алгоритм Левенштейна работает на несколько порядков медленнее матричной реализации. Рекурсивный алгоритм с параллельным заполнением матрицы работает быстрее простого рекурсивного, но все еще медленнее матричного. Если длина сравниваемых строк превышает 10, рекурсивный алгоритм становится неприемлимым для использования по времени выполнения программы. Матричная реализация алгоритма Дамерау — Левенштейна сопоставима с алгоритмом Левенштейна. В ней добавлены дополнительные проверки, что делает его немного медленнее.

Теоретически было рассчитано использования памяти в каждом из алгоритмов нахождения расстояния Левенштейна. Обычные матричные алгоритмы потребляют намного больше памяти, чем рекурсивные, за счет дополнительного выделения памяти под матрицы.
