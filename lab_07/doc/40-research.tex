\chapter{Исследовательская часть}

В данном разделе приведены примеры работы программы и анализ характериситик разработанного программного обеспечения.

\section{Технические характеристики}

\begin{itemize}
    \item Операционная система: Manjaro \cite{manjaro} Linux \cite{linux} x86\_64.
    \item Память: 8 ГБ.
    \item Процессор: Intel® Core™ i7-8550U\cite{intel}.
\end{itemize}

Тестирование проводилось на ноутбуке, включенном в сеть электропитания. Во время тестирования ноутбук был нагружен только встроенными приложениями окружения, окружением, а также непосредственно системой тестирования.

\section{Замеры и исследование результатов.}

Тестирование проводилось на 2 наборах для каждого алгоритма и каждого количества подборов. Первый набор включал в себя исключительно ключи, которые есть в словаре, причём данные ключи перебираются итеративно, что имитирует равновозможность выпадения ключа (в тестовом наборе 2409 ключей, таким образом для 10.000.000 подборов каждый ключ берется $\approx 4151$ раз). Второй набор включал исключительно отсутствующие в словаре ключи, что позволяло проверить, насколько быстро алгоритмы поиска способны обнаружить тот факт, что ключа нет среди имеющихся.

В таблицах \ref{tbl:brute}, \ref{tbl:binary} и \ref{tbl:segment} представлены времена работы полного перебора, двоичного поиска и сегментированного алгоритмов. В таблице представлены значения:
\begin{itemize}
    \item количество раз, которое производился поиск (Кол-во);
    \item суммарное время поиска только присутствующих ключей (ВППК; в нс);
    \item суммарное время поиска только отсутствующих ключей (ВППК; в нс).
\end{itemize}

Примечание: для сегментированного алгоритма было выбрано деление на 5 сегментов.

\begin{table}[h!]
    \begin{center}
        \begin{tabular}{|c|c|c|}
            \hline
            Кол-во & ВППК & ВПОК
            \csvreader{assets/csv/brute.csv}{}
            {\\\hline \csvcoli&\csvcolii&\csvcoliii}
            \\\hline
        \end{tabular}
    \end{center}
    \caption{Время работы полного перебора.}
    \label{tbl:brute}
\end{table}

\begin{table}[h!]
    \begin{center}
        \begin{tabular}{|c|c|c|}
            \hline
            Кол-во & ВППК & ВПОК
            \csvreader{assets/csv/binary.csv}{}
            {\\\hline \csvcoli&\csvcolii&\csvcoliii}
            \\\hline
        \end{tabular}
    \end{center}
    \caption{Время работы двоичного поиска.}
    \label{tbl:binary}
\end{table}

\begin{table}[h!]
    \begin{center}
        \begin{tabular}{|c|c|c|}
            \hline
            Кол-во & ВППК & ВПОК
            \csvreader{assets/csv/segment.csv}{}
            {\\\hline \csvcoli&\csvcolii&\csvcoliii}
            \\\hline
        \end{tabular}
    \end{center}
    \caption{Время работы сегментированного алгоритма с частотным анализом.}
    \label{tbl:segment}
\end{table}

На графике \ref{plt:time} построены графики зависимости времени от количества подборов из словаря для алгоритмов поиска (для данных из таблиц \ref{tbl:brute}, \ref{tbl:binary} и \ref{tbl:segment}):
\begin{itemize}
    \item полного перебора при запросе только присутствующих ключей (BruteGood);
    \item полного перебора при запросе только отсутствующих ключей (BruteBad);
    \item бинарного при запросе только присутствующих ключей (BinaryGood);
    \item бинарного при запросе только отсутствующих ключей (BinaryBad);
    \item сегментированного с частотным анализом при запросе только присутствующих ключей (SegmentGood);
    \item сегментированного с частотным анализом при запросе только отсутствующих ключей (SegmentBad).
\end{itemize}



\begin{figure}[h]
    \centering
    \begin{tikzpicture}
        \begin{axis}[
                axis lines=left,
                xlabel=Количество подборов,
                legend pos=north west,
                ylabel={Время, нс},
                ymajorgrids=true]
            \addplot table[x=amount,y=good,col sep=comma] {assets/csv/brute.csv};
            \addplot table[x=amount,y=bad,col sep=comma] {assets/csv/brute.csv};
            \addplot table[x=amount,y=good,col sep=comma] {assets/csv/binary.csv};
            \addplot table[x=amount,y=bad,col sep=comma] {assets/csv/binary.csv};
            \addplot table[x=amount,y=good,col sep=comma] {assets/csv/segment.csv};
            \addplot table[x=amount,y=bad,col sep=comma] {assets/csv/segment.csv};
            \legend{BruteGood, BruteBad, BinaryGood, BinaryBad, SegmentGood, SegmentBad}
        \end{axis}
    \end{tikzpicture}
    \captionsetup{justification=centering}
    \caption{Сравнение алгоритмов.}
    \label{plt:time}
\end{figure}


\section*{Вывод}

Алгоритм полного перебора оказался самым медленным, при 10.000.000 элементах и поиске только присутствующих ключей среднее время поиска $\approx 34.121$ мкс, в то время как среднее время поиска сегментированного алгоритма на тех же данных составило $\approx 0.406$ мкс, что приблизительно равно времени бинарного поиска $\approx 0.405$ мкс, который оказался менее чем на 1\% быстрее. При том же наборе, но поиске только отсутствующих ключей полный перебор работает медленнее на $\approx 35.474$ мкс (что составляет примерно 4\%), в то время как времена сегментированного алгоритма ($\approx 0.408$) и бинарного поиска ($\approx 0.406$), также ухудшились, но менее, чем на 1\%.
