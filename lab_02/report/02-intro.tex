\chapter*{Введение}
\addcontentsline{toc}{chapter}{Введение}

Алгоритм Копперсмита — Винограда — алгоритм умножения квадратных матриц, предложенный в 1987 году Д. Копперсмитом и Ш. Виноградом \cite{Coppersmith}.
В исходной версии асимптотическая сложность алгоритма составляла $O(n^{2,3755})$, где  $n$ — размер стороны матрицы.
Алгоритм Копперсмита — Винограда, с учетом серии улучшений и доработок в последующие годы, обладает лучшей асимптотикой среди известных алгоритмов умножения матриц \cite{Cohn}.

На практике алгоритм Копперсмита — Винограда не используется, так как он имеет очень большую константу пропорциональности и начинает выигрывать в быстродействии у других известных алгоритмов только для матриц, размер которых превышает память современных компьютеров \cite{Robinson}.
Поэтому пользуются алгоритмом Штрассена по причинам простоты реализации и меньшей константе в оценке трудоемкости.

Алгоритм Штрассена \cite{Strassen} предназначен для быстрого умножения матриц.
Он был разработан Фолькером Штрассеном в 1969 году и является обобщением метода умножения Карацубы на матрицы.

В отличие от традиционного алгоритма умножения матриц, алгоритм Штрассена умножает матрицы за время ${\displaystyle \Theta (n^{\log _{2}7})=O(n^{2.81})}$, что даёт выигрыш на больших плотных матрицах начиная, примерно, от $64\times64$.

Несмотря на то, что алгоритм Штрассена является асимптотически не самым быстрым из существующих алгоритмов быстрого умножения матриц, он проще программируется и эффективнее при умножении матриц относительно малого размера.

Задачи лабораторной работы:

\begin{enumerate}
	\item изучение и реализация 3 алгоритмов перемножения матриц: обычный, Копперсмита-Винограда, улучшенный Копперсмита-Винограда;
	\item сравнительный анализ алгоритмов на основе теоретических расчетов трудоемкости в выбранной модели вычислений;
	\item сравнительный анализ алгоритмов на основе экспериментальных данных;
    \item подготовка отчета по лабораторной работе.
\end{enumerate}

