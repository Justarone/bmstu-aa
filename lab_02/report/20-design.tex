\chapter{Конструкторская часть}

\section{Разработка алгоритмов}

На рисунке \ref{img:simple} приведена схема стандартного алгоритма умножения матриц.

На рисунках \ref{img:vino} и \ref{img:mv_mh} представлена схема алгоритма Копперсмита — Винограда и его функций заполения сумм строк и столбцов.

На рисунках \ref{img:vino_improved} и \ref{img:mv_mh_fast} представлена схема улучшенного алгоритма Копперсмита — Винограда и его функций заполения сумм строк и столбцов.

%\clearpage
\img{170mm}{simple}{Стандартный алгоритм заполнения матриц}

На схеме видно, что для стандартного алгоритма не существует лучшего и худшего случаев как таковых ввиду отсутствия ветвлений.

\img{170mm}{vino}{Cхема алгоритма Копперсмита — Винограда}
\clearpage
\img{170mm}{mv_mh}{Cхемы функций алгоритма Копперсмита — Винограда}
Пусть в дальнейшем K - общий размер при умножении матриц размеров $M \times K$ и $K \times N$.

Видно, что для алгоритма Винограда худшим случаем являются матрицы с нечётным общим размером, а лучшим - с чётным, т. к. отпадает необходимость в последнем цикле.

Данный алгоритм можно оптимизировать:
\begin{itemize}
	\item заменой операции деления на 2 побитовым сдвигом на 1 вправо;
	\item заменой выражения вида \code{a = a + ...} на \code{a += ...};
	\item сделав в циклах по k шаг 2, избавившись тем самым от двух операций умножения на каждую итерацию.
\end{itemize}

\img{170mm}{vino_improved}{Cхема алгоритма улучшенного Копперсмита — Винограда}
\clearpage
\img{170mm}{mv_mh_fast}{Cхемы функций алгоритма улучшенного Копперсмита — Винограда}


\section{Модель вычислений}

Для последующего вычисления трудоемкости необходимо ввести модель вычислений:
\begin{enumerate}
    \item операции из списка (\ref{for:opers}) имеют трудоемкость 1;
        \begin{equation}
            \label{for:opers}
            +, -, /, \%, ==, !=, <, >, <=, >=, [], ++, {-}-
        \end{equation}
    \item трудоемкость оператора выбора \code{if условие then A else B} рассчитывается, как (\ref{for:if});
	\begin{equation}
        \label{for:if}
        f_{if} = f_{\text{условия}} +
        \begin{cases}
        f_A, & \text{если условие выполняется,}\\
        f_B, & \text{иначе.}
        \end{cases}
	\end{equation}
\item трудоемкость цикла рассчитывается, как (\ref{for:for});
    \begin{equation}
        \label{for:for}
        f_{for} = f_{\text{инициализации}} + f_{\text{сравнения}} + N(f_{\text{тела}} + f_{\text{инициализации}} + f_{\text{сравнения}})
    \end{equation}
	\item трудоемкость вызова функции равна 0.
\end{enumerate}

\section{Трудоемкость алгоритмов}

\subsection{Стандартный алгоритм умножения матриц}

Во всех последующих алгоритмах не будем учитывать инициализацию матрицу, в которую записывается результат, потому что данное действие есть во всех алгоритмах и при этом не является самым трудоёмким.

Трудоёмкость стандартного алгоритма умножения матриц состоит из:
\begin{itemize}
    \item внешнего цикла по $i \in [1..M]$, трудоёмкость которого: $f = 2 + M \cdot (2 + f_{body})$;
    \item цикла по $j \in [1..N]$, трудоёмкость которого: $f = 2 + N \cdot (2 + f_{body})$;
    \item цикла по $k \in [1..K]$, трудоёмкость которого: $f = 2 + 10K$;
\end{itemize}

Учитывая, что трудоёмкость стандартного алгоритма равна трудоёмкости внешнего цикла, можно вычислить ее, подставив циклы тела (\ref{for:standard}):
\begin{equation}
    \label{for:standard}
    f_{standard} = 2 + M \cdot (4 + N \cdot (4 + 10K)) = 2 + 4M + 4MN + 10MNK \approx 10MNK
\end{equation}

\subsection{Алгоритм Копперсмита — Винограда}

Трудоёмкость алгоритма Копперсмита — Винограда состоит из:
\begin{enumerate}
    \item создания и инициализации массивов MH и MV, трудоёмкость которого (\ref{for:init}):
        \begin{equation}
            \label{for:init}
        f_{init} = M + N;
        \end{equation}

    \item заполнения массива MH, трудоёмкость которого (\ref{for:MH}):
        \begin{equation}
            \label{for:MH}
        f_{MH} = 3 + \frac{K}{2} \cdot (5 + 12M);
        \end{equation}

    \item заполнения массива MV, трудоёмкость которого (\ref{for:MV}):
        \begin{equation}
            \label{for:MV}
        f_{MV} = 3 + \frac{K}{2} \cdot (5 + 12N);
        \end{equation}

    \item цикла заполнения для чётных размеров, трудоёмкость которого (\ref{for:cycle}):
        \begin{equation}
            \label{for:cycle}
        f_{cycle} = 2 + M \cdot (4 + N \cdot (11 + \frac{25}{2} \cdot K));
        \end{equation}

    \item цикла, для дополнения умножения суммой последних нечётных строки и столбца, если общий размер нечётный, трудоемкость которого (\ref{for:last}):
        \begin{equation}
            \label{for:last}
            f_{last} = \begin{cases}
                2, & \text{чётная,}\\
                4 + M \cdot (4 + 14N), & \text{иначе.}
            \end{cases}
        \end{equation}
\end{enumerate}

Итого, для худшего случая (нечётный общий размер матриц) имеем (\ref{for:bad}):
\begin{equation}
    \label{for:bad}
    f = M + N + 12 + 8M + 5K + 6MK + 6NK + 25MN + \frac{25}{2}MNK \approx 12.5 \cdot MNK
\end{equation}

Для лучшего случая (чётный общий размер матриц) имеем (\ref{for:good}):
\begin{equation}
    \label{for:good}
    f = M + N + 10 + 4M + 5K + 6MK + 6NK + 11MN + \frac{25}{2}MNK \approx 12.5 \cdot MNK
\end{equation}

\subsection{Оптимизированный алгоритм Копперсмита — Винограда}

Трудоёмкость улучшенного алгоритма Копперсмита — Винограда состоит из:
\begin{enumerate}
    \item создания и инициализации массивов MH и MV, трудоёмкость которого (\ref{for:impr_init}):
        \begin{equation}
            \label{for:impr_init}
            f_{init} = M + N;
        \end{equation}

    \item заполнения массива MH, трудоёмкость которого (\ref{for:impr_MH}):
        \begin{equation}
            \label{for:impr_MH}
            f_{MH} = 2 + \frac{K}{2} \cdot (4 + 8M);
        \end{equation}

    \item заполнения массива MV, трудоёмкость которого (\ref{for:impr_MV}):
        \begin{equation}
            \label{for:impr_MV}
            f_{MV} = 2 + \frac{K}{2} \cdot (4 + 8N);
        \end{equation}

    \item цикла заполнения для чётных размеров, трудоёмкость которого (\ref{for:impr_cycle}):
        \begin{equation}
            \label{for:impr_cycle}
            f_{cycle} = 2 + M \cdot (4 + N \cdot (8 + 9K))
        \end{equation}

    \item цикла, для дополнения умножения суммой последних нечётных строки и столбца, если общий размер нечётный, трудоемкость которого (\ref{for:impr_last}):
        \begin{equation}
            \label{for:impr_last}
                f_{last} = 
                \begin{cases}
                    2, & \text{чётная,}\\
                    4 + M \cdot (4 + 12N), & \text{иначе.}
            \end{cases}
        \end{equation}
\end{enumerate}

Итого, для худшего случая (нечётный общий размер матриц) имеем (\ref{for:bad_impr}):
\begin{equation}
    \label{for:bad_impr}
    f = M + N + 10 + 4K + 4KN + 4KM + 8M + 20MN + 9MNK \approx 9MNK
\end{equation}

Для лучшего случая (чётный общий размер матриц) имеем (\ref{for:good_impr}):
\begin{equation}
    \label{for:good_impr}
    f = M + N + 8 + 4K +4KN + 4KM + 4M + 8MN + 9MNK \approx 9MNK
\end{equation}


\section*{Вывод}

На основе теоретических данных, полученных из аналитического раздела, были построены схемы обоих алгоритмов умножения матриц.  Оценены их трудоёмкости в лучшем и худшем случаях.
