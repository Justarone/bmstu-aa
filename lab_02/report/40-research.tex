\chapter{Исследовательская часть}

\section{Технические характеристики}

\begin{itemize}
	\item Операционная система: Manjaro \cite{manjaro} Linux \cite{linux} x86\_64.
	\item Память: 8 GiB.
	\item Процессор: Intel® Core™ i7-8550U\cite{intel}.
\end{itemize}

Тестирование проводилось на ноутбуке, включенном в сеть электропитания. Во время тестирования ноутбук был нагружен только встроенными приложениями окружения, окружением, а также непосредственно системой тестирования.

\section{Время выполнения алгоритмов}

Результаты замеров приведены в таблицах \ref{tbl:test_even} и \ref{tbl:test_odd}.
На рисунках \ref{plt:test_even} и \ref{plt:test_odd} приведены графики зависимостей времени работы алгоритмов от размеров матриц. Здесь и далее: С — стандартный алгоритм, КВ — алгоритм Копперсмита — Винограда, УКВ — улучшенный алгоритм Копперсмита — Винограда.

\begin{table}[h!]
	\begin{center}
		\begin{tabular}{|c|c|c|c|}
			\hline
			                         & \multicolumn{3}{c|}{\bfseries Время, нс}      \\ \cline{2-4}
			\bfseries Размер матрицы & \bfseries С & \bfseries КВ & \bfseries УКВ
			\csvreader{assets/csv/test-even.csv}{}
			{\\\hline \csvcoli&\csvcolii&\csvcoliii&\csvcoliv}
			\\\hline
		\end{tabular}
	\end{center}
	\caption{Время работы алгоритмов при чётных размерах матриц}
	\label{tbl:test_even}
\end{table}

\begin{table}
	\begin{center}
		\begin{tabular}{|c|c|c|c|}
			\hline
			                         & \multicolumn{3}{c|}{\bfseries Время, нс}      \\ \cline{2-4}
			\bfseries Размер матрицы & \bfseries С & \bfseries КВ & \bfseries УКВ
			\csvreader{assets/csv/test-odd.csv}{}
			{\\\hline \csvcoli&\csvcolii&\csvcoliii&\csvcoliv}
			\\\hline
		\end{tabular}
	\end{center}
	\caption{Время работы алгоритмов при нечётных размерах матриц}
	\label{tbl:test_odd}
\end{table}

\begin{figure}
	\centering
	\begin{tikzpicture}
		\begin{axis}[
			axis lines=left,
			xlabel=Размер,
			ylabel={Время, мс},
			legend pos=north west,
			ymajorgrids=true
		]
			\addplot table[x=size,y=product,col sep=comma]{assets/csv/test-even.csv};
			\addplot table[x=size,y=bad_winograd,col sep=comma]{assets/csv/test-even.csv};
			\addplot table[x=size,y=good_winograd,col sep=comma]{assets/csv/test-even.csv};
			\legend{С, КВ, УКВ}
		\end{axis}
	\end{tikzpicture}
	\captionsetup{justification=centering}
	\caption{Зависимость времени работы алгоритма от чётного размера квадратной матрицы}
	\label{plt:test_even}
\end{figure}

\begin{figure}
	\centering
	\begin{tikzpicture}
		\begin{axis}[
			axis lines=left,
			xlabel=Размер,
			ylabel={Время, мс},
			legend pos=north west,
			ymajorgrids=true
		]
			\addplot table[x=size,y=product,col sep=comma]{assets/csv/test-odd.csv};
			\addplot table[x=size,y=bad_winograd,col sep=comma]{assets/csv/test-odd.csv};
			\addplot table[x=size,y=good_winograd,col sep=comma]{assets/csv/test-odd.csv};
			\legend{С, КВ, УКВ}
		\end{axis}
	\end{tikzpicture}
	\captionsetup{justification=centering}
	\caption{Зависимость времени работы алгоритма от нечётного размера квадратной матрицы}
	\label{plt:test_odd}
\end{figure}

\section*{Вывод}

Время работы алгоритма Винограда незначительно меньше стандартного алгоритма умножения, однако при размере $800 \times 800$ время вычислений алгоритма Винограда на 0.3 секунды меньше, нежели у стандартного алгоритма (что составляет в данном случае примерно 10\%).
