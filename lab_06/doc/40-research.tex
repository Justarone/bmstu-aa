\chapter{Исследовательская часть}

В данном разделе приведены примеры работы программы и анализ характериситик разработанного программного обеспечения.

\section{Технические характеристики}

\begin{itemize}
    \item Операционная система: Manjaro \cite{manjaro} Linux \cite{linux} x86\_64.
    \item Память: 8 ГБ.
    \item Процессор: Intel® Core™ i7-8550U\cite{intel}.
\end{itemize}

Тестирование проводилось на ноутбуке, включенном в сеть электропитания. Во время тестирования ноутбук был нагружен только встроенными приложениями окружения, окружением, а также непосредственно системой тестирования.

\section{Пример работы программы}
На рисунке \ref{img:example} приведен пример работы программы.
\img{100mm}{example}{Пример работы программы.}
\clearpage


\section{Исследование скорости работы алгоритмов.}
Алгоритмы тестировались на данных, сгенерированных случайным образом один раз.

Результаты замеров времени приведены в таблице \ref{tab:time}. На рисунке \ref{plt:time} приведен графики зависимостей времени работы алгоритмов от размерности матрицы смежности.


\begin{table}[h!]
    \begin{center}
        \begin{tabular}{|c|c|c|}
            \hline
            Размерность матрицы & Полный перебор & Муравьиный алгоритм \\
            \csvreader{assets/csv/time.csv}{}
            {\\\hline \csvcoli&\csvcolii&\csvcoliii}
            \\\hline
        \end{tabular}
    \end{center}
    \caption{Замеры времени}
    \label{tab:time}
\end{table}

\begin{figure}[h]
    \centering
    \begin{tikzpicture}
        \begin{axis}[
                axis lines=left,
                xlabel=Размер матрицы,
                legend pos=north west,
                ylabel={Время, нс},
                ymajorgrids=true]
            \addplot table[x=size,y=brute,col sep=comma] {assets/csv/time.csv};
            \addplot table[x=size,y=ants,col sep=comma] {assets/csv/time.csv};
            \legend{Brute, Ants}
        \end{axis}
    \end{tikzpicture}
    \captionsetup{justification=centering}
    \caption{График времени работы алгоритмов.}
    \label{plt:time}
\end{figure}

\section{Постановка эксперимента}
В муравьином алгоритме вычисления производятся на основе настраиваемых параметров.
Рассмотрим два класса данных и подберем к ним параметры, при которых метод даст точный результат при небольшом количестве итераций.

Будем рассматривать матрицы размерности $10\times10$, так как иначе получение точного результата алгоритмом велико и менее зависимо от параметров.

В качестве первого класса данных выделим матрицу смежности, в которой все значения незначительно отличаются друг от друга, например, в диапазоне $[1, 25]$.
Вторым классом будут матрицы, где значения могут значительно отличаться, например $[1, 2500]$.

Будем запускать муравьиный алгоритм для всех значений $\alpha, \rho\in[0, 1]$, с шагом $= 0.1$, пока не будет найдено точное значение для каждого набора.

В результате тестирования будет выведена таблица со значениями $\alpha, \beta, \rho, Длина, Разница$, где $Длина$ — оптимальная длина пути, найденная муравьиным алгоритмом, $Разница$ - Разность между полученным значением и настоящей оптимальной длиной пути, а $\alpha, \beta, \rho$ — настраиваемые параметры.

\subsection{Код тестирования}

В листинге \ref{lst:tests} представлен код, с помощью которого проводилась параметризация.

\begin{lstinputlisting}[
        caption={Код для параметризации.},
        label={lst:tests},
        style={rust},
        linerange={34-80}
    ]{../src/lib/mod.rs}
\end{lstinputlisting}


\subsection{Класс данных 1}
\begin{equation}
    \label{matrix}
    M = \begin{pmatrix}
        0 &    21 &    21 &     4 &    22 &    20 &     6 &     4 &    11 &    23 \\
        21 &     0 &    22 &     6 &     9 &    21 &    16 &     6 &     2 &     7 \\
        21 &    22 &     0 &    18 &    23 &     7 &    19 &    19 &     1 &    14 \\
        4 &     6 &    18 &     0 &     7 &    11 &    16 &    10 &     8 &    19 \\
        22 &     9 &    23 &     7 &     0 &    15 &     2 &    23 &     1 &    22 \\
        20 &    21 &     7 &    11 &    15 &     0 &     6 &     7 &     8 &    21 \\
        6 &    16 &    19 &    16 &     2 &     6 &     0 &     8 &     3 &    14 \\
        4 &     6 &    19 &    10 &    23 &     7 &     8 &     0 &    23 &    15 \\
        11 &     2 &     1 &     8 &     1 &     8 &     3 &    23 &     0 &     9 \\
        23 &     7 &    14 &    19 &    22 &    21 &    14 &    15 &     9 &     0 \\
    \end{pmatrix}
\end{equation}

В таблице ~\ref{tab:log1} приведены результаты параметризации метода решения задачи коммивояжера на основании муравьиного алгоритма. Количество дней было взято равным 50. Полный перебор определил оптимальную длину пути 52. Два последних стобца таблицы определяют найденный муравьиным алгоритм оптимальный путь и разницу этого пути с оптимальным путём, найденным алгоритмом полного перебора.

\begin{table}
    \caption{Таблица коэффициентов для класса данных 1.}
    \begin{minipage}[h!]{0.10\hsize}\centering
        \begin{center}\resizebox{4\textwidth}{!}{%
                \begin{tabular}{|c@{\hspace{5mm}}|c@{\hspace{5mm}}|c@{\hspace{5mm}}|c@{\hspace{5mm}}|c@{\hspace{5mm}}|c|}
                    \hline
                    \alpha        & \beta      & \rho      &Длина  & Разница \\
                    \hline
                    0    & 1    & 0    & 52    & 0     \\
                    0    & 1    & 0.1  & 52    & 0     \\
                    0    & 1    & 0.2  & 52    & 0     \\
                    0    & 1    & 0.3  & 52    & 0     \\
                    0    & 1    & 0.4  & 53    & 1     \\
                    0    & 1    & 0.5  & 52    & 0     \\
                    0    & 1    & 0.6  & 52    & 0     \\
                    0    & 1    & 0.7  & 52    & 0     \\
                    0    & 1    & 0.8  & 52    & 0     \\
                    0    & 1    & 0.9  & 52    & 0     \\
                    0    & 1    & 1    & 52    & 0     \\
                    0.1  & 0.9  & 0    & 52    & 0     \\
                    0.1  & 0.9  & 0.1  & 52    & 0     \\
                    0.1  & 0.9  & 0.2  & 53    & 1     \\
                    0.1  & 0.9  & 0.3  & 52    & 0     \\
                    0.1  & 0.9  & 0.4  & 52    & 0     \\
                    0.1  & 0.9  & 0.5  & 52    & 0     \\
                    0.1  & 0.9  & 0.6  & 53    & 1     \\
                    0.1  & 0.9  & 0.7  & 52    & 0     \\
                    0.1  & 0.9  & 0.8  & 52    & 0     \\
                    0.1  & 0.9  & 0.9  & 52    & 0     \\
                    0.1  & 0.9  & 1    & 52    & 0     \\
                    0.2  & 0.8  & 0    & 52    & 0     \\
                    0.2  & 0.8  & 0.1  & 52    & 0     \\
                    0.2  & 0.8  & 0.2  & 52    & 0     \\
                    0.2  & 0.8  & 0.3  & 53    & 1     \\
                    0.2  & 0.8  & 0.4  & 52    & 0     \\
                    0.2  & 0.8  & 0.5  & 52    & 0     \\
                    0.2  & 0.8  & 0.6  & 53    & 1     \\
                    0.2  & 0.8  & 0.7  & 53    & 1     \\
                    0.2  & 0.8  & 0.8  & 52    & 0     \\
                    0.2  & 0.8  & 0.9  & 52    & 0     \\
                    0.2  & 0.8  & 1    & 52    & 0     \\
                    0.3  & 0.7  & 0    & 53    & 1     \\
                    0.3  & 0.7  & 0.1  & 52    & 0     \\
                    0.3  & 0.7  & 0.2  & 52    & 0     \\
                    0.3  & 0.7  & 0.3  & 53    & 1     \\
                    0.3  & 0.7  & 0.4  & 52    & 0     \\
                    0.3  & 0.7  & 0.5  & 52    & 0     \\
                    0.3  & 0.7  & 0.6  & 52    & 0     \\
                \end{tabular}}
                \label{tab:log1}
        \end{center}
    \end{minipage}
    \hfill
    \begin{minipage}[!h]{0.50\hsize}\centering
        \begin{center}\resizebox{0.8\textwidth}{!}{%
                %\caption{Лог работы программы.}
                \begin{tabular}{|c@{\hspace{5mm}}|c@{\hspace{5mm}}|c@{\hspace{5mm}}|c@{\hspace{5mm}}|c@{\hspace{5mm}}|c|}
                    \alpha        & \beta      & \rho      &Длина  & Разница \\
                    \hline
                    0.3  & 0.7  & 0.7  & 52    & 0     \\
                    0.3  & 0.7  & 0.8  & 52    & 0     \\
                    0.3  & 0.7  & 0.9  & 53    & 1     \\
                    0.3  & 0.7  & 1    & 52    & 0     \\
                    0.4  & 0.6  & 0    & 53    & 1     \\
                    0.4  & 0.6  & 0.1  & 53    & 1     \\
                    0.4  & 0.6  & 0.2  & 53    & 1     \\
                    0.4  & 0.6  & 0.3  & 52    & 0     \\
                    0.4  & 0.6  & 0.4  & 53    & 1     \\
                    0.4  & 0.6  & 0.5  & 52    & 0     \\
                    0.4  & 0.6  & 0.6  & 52    & 0     \\
                    0.4  & 0.6  & 0.7  & 52    & 0     \\
                    0.4  & 0.6  & 0.8  & 52    & 0     \\
                    0.4  & 0.6  & 0.9  & 52    & 0     \\
                    0.4  & 0.6  & 1    & 52    & 0     \\
                    0.5  & 0.5  & 0    & 52    & 0     \\
                    0.5  & 0.5  & 0.1  & 52    & 0     \\
                    0.5  & 0.5  & 0.2  & 52    & 0     \\
                    0.5  & 0.5  & 0.3  & 53    & 1     \\
                    0.5  & 0.5  & 0.4  & 52    & 0     \\
                    0.5  & 0.5  & 0.5  & 52    & 0     \\
                    0.5  & 0.5  & 0.6  & 52    & 0     \\
                    0.5  & 0.5  & 0.7  & 52    & 0     \\
                    0.5  & 0.5  & 0.8  & 52    & 0     \\
                    0.5  & 0.5  & 0.9  & 52    & 0     \\
                    0.5  & 0.5  & 1    & 52    & 0     \\
                    0.6  & 0.4  & 0    & 53    & 1     \\
                    0.6  & 0.4  & 0.1  & 53    & 1     \\
                    0.6  & 0.4  & 0.2  & 56    & 4     \\
                    0.6  & 0.4  & 0.3  & 52    & 0     \\
                    0.6  & 0.4  & 0.4  & 52    & 0     \\
                    0.6  & 0.4  & 0.5  & 55    & 3     \\
                    0.6  & 0.4  & 0.6  & 56    & 4     \\
                    0.6  & 0.4  & 0.7  & 52    & 0     \\
                    0.6  & 0.4  & 0.8  & 53    & 1     \\
                    0.6  & 0.4  & 0.9  & 53    & 1     \\
                    0.6  & 0.4  & 1    & 52    & 0     \\
                    0.7  & 0.3  & 0    & 52    & 0     \\
                    0.7  & 0.3  & 0.1  & 53    & 1     \\
                    0.7  & 0.3  & 0.2  & 52    & 0     \\
                \end{tabular}}
        \end{center}
    \end{minipage}
\end{table}
\clearpage
\begin{table}[!h]
    \begin{center}
        \begin{tabular}{|c@{\hspace{7mm}}|c@{\hspace{7mm}}|c@{\hspace{7mm}}|c@{\hspace{7mm}}|c@{\hspace{7mm}}|c|}
            \alpha        & \beta      & \rho      &Длина  & Разница \\
            \hline
            0.7  & 0.3  & 0.3  & 52    & 0     \\
            0.7  & 0.3  & 0.4  & 53    & 1     \\
            0.7  & 0.3  & 0.5  & 53    & 1     \\
            0.7  & 0.3  & 0.6  & 52    & 0     \\
            0.7  & 0.3  & 0.7  & 53    & 1     \\
            0.7  & 0.3  & 0.8  & 57    & 5     \\
            0.7  & 0.3  & 0.9  & 52    & 0     \\
            0.7  & 0.3  & 1    & 52    & 0     \\
            0.8  & 0.2  & 0    & 59    & 7     \\
            0.8  & 0.2  & 0.1  & 53    & 1     \\
            0.8  & 0.2  & 0.2  & 56    & 4     \\
            0.8  & 0.2  & 0.3  & 53    & 1     \\
            0.8  & 0.2  & 0.4  & 52    & 0     \\
            0.8  & 0.2  & 0.5  & 56    & 4     \\
            0.8  & 0.2  & 0.6  & 53    & 1     \\
            0.8  & 0.2  & 0.7  & 52    & 0     \\
            0.8  & 0.2  & 0.8  & 52    & 0     \\
            0.8  & 0.2  & 0.9  & 52    & 0     \\
            0.8  & 0.2  & 1    & 53    & 1     \\
            0.9  & 0.1  & 0    & 56    & 4     \\
            0.9  & 0.1  & 0.1  & 53    & 1     \\
            0.9  & 0.1  & 0.2  & 52    & 0     \\
            0.9  & 0.1  & 0.3  & 56    & 4     \\
            0.9  & 0.1  & 0.4  & 52    & 0     \\
            0.9  & 0.1  & 0.5  & 53    & 1     \\
            0.9  & 0.1  & 0.6  & 56    & 4     \\
            0.9  & 0.1  & 0.7  & 56    & 4     \\
            0.9  & 0.1  & 0.8  & 55    & 3     \\
            0.9  & 0.1  & 0.9  & 53    & 1     \\
            0.9  & 0.1  & 1    & 53    & 1     \\
            1    & 0    & 0    & 71    & 19    \\
            1    & 0    & 0.1  & 61    & 9     \\
            1    & 0    & 0.2  & 53    & 1     \\
            1    & 0    & 0.3  & 59    & 7     \\
            1    & 0    & 0.4  & 59    & 7     \\
            1    & 0    & 0.5  & 60    & 8     \\
            1    & 0    & 0.6  & 60    & 8     \\
            1    & 0    & 0.7  & 74    & 22    \\
            1    & 0    & 0.8  & 60    & 8     \\
            1    & 0    & 0.9  & 57    & 5     \\
            1    & 0    & 1    & 60    & 8     \\
            \hline
        \end{tabular}
    \end{center}
\end{table}
\clearpage

\subsection{Класс данных 2}

\begin{equation}
    \label{matrix1}
    M = \begin{pmatrix}
        0 &  1790 &   200 &  1900 &    63 &  1659 &  1820 &  1395 &  2382 &   649 \\
        1790 &     0 &  1573 &  2435 &  1515 &   714 &   892 &  2193 &  1590 &  1003 \\
        200 &  1573 &     0 &   833 &   392 &  2404 &   962 &   902 &   141 &  1123 \\
        1900 &  2435 &   833 &     0 &  2283 &  1652 &  2362 &  2262 &  1512 &  2166 \\
        63 &  1515 &   392 &  2283 &     0 &  1322 &   290 &  1305 &  2100 &   969 \\
        1659 &   714 &  2404 &  1652 &  1322 &     0 &   256 &    78 &  2236 &  2041 \\
        1820 &   892 &   962 &  2362 &   290 &   256 &     0 &  1180 &  1547 &  1279 \\
        1395 &  2193 &   902 &  2262 &  1305 &    78 &  1180 &     0 &  1640 &  1161 \\
        2382 &  1590 &   141 &  1512 &  2100 &  2236 &  1547 &  1640 &     0 &  2212 \\
        649 &  1003 &  1123 &  2166 &   969 &  2041 &  1279 &  1161 &  2212 &     0 \\
    \end{pmatrix}
\end{equation}


В таблице ~\ref{tab:log2} приведены результаты параметризации метода решения задачи коммивояжера на основании муравьиного алгоритма для матрицы с элементами в диапазоне $[0, 2500]$. Количество дней было взято равным 50. Полный перебор определил оптимальную длину пути 6986. Два последних стобца таблицы определяют найденный муравьиным алгоритм оптимальный путь и разницу этого пути с оптимальным путём, найденным алгоритмом полного перебора.

\begin{table}
    \caption{Таблица коэффициентов для класса данных №2}
    \begin{minipage}[h!]{0.10\hsize}\centering
        \begin{center}\resizebox{4\textwidth}{!}{%
                \begin{tabular}{|c@{\hspace{5mm}}|c@{\hspace{5mm}}|c@{\hspace{5mm}}|c@{\hspace{5mm}}|c@{\hspace{5mm}}|c|}
                    \hline
                    \alpha        & \beta      & \rho      &Длина  & Разница \\
                    \hline
                    0    & 1    & 0    & 6986  & 0     \\
                    0    & 1    & 0.1  & 6986  & 0     \\
                    0    & 1    & 0.2  & 6986  & 0     \\
                    0    & 1    & 0.3  & 6986  & 0     \\
                    0    & 1    & 0.4  & 6986  & 0     \\
                    0    & 1    & 0.5  & 6986  & 0     \\
                    0    & 1    & 0.6  & 6986  & 0     \\
                    0    & 1    & 0.7  & 6986  & 0     \\
                    0    & 1    & 0.8  & 6986  & 0     \\
                    0    & 1    & 0.9  & 6992  & 6     \\
                    0    & 1    & 1    & 6986  & 0     \\
                    0.1  & 0.9  & 0    & 6986  & 0     \\
                    0.1  & 0.9  & 0.1  & 6992  & 6     \\
                    0.1  & 0.9  & 0.2  & 6986  & 0     \\
                    0.1  & 0.9  & 0.3  & 6986  & 0     \\
                    0.1  & 0.9  & 0.4  & 6986  & 0     \\
                    0.1  & 0.9  & 0.5  & 6986  & 0     \\
                    0.1  & 0.9  & 0.6  & 6986  & 0     \\
                    0.1  & 0.9  & 0.7  & 6986  & 0     \\
                    0.1  & 0.9  & 0.8  & 6986  & 0     \\
                    0.1  & 0.9  & 0.9  & 7165  & 179   \\
                    0.1  & 0.9  & 1    & 6986  & 0     \\
                    0.2  & 0.8  & 0    & 6986  & 0     \\
                    0.2  & 0.8  & 0.1  & 6986  & 0     \\
                    0.2  & 0.8  & 0.2  & 6986  & 0     \\
                    0.2  & 0.8  & 0.3  & 6992  & 6     \\
                    0.2  & 0.8  & 0.4  & 6992  & 6     \\
                    0.2  & 0.8  & 0.5  & 6992  & 6     \\
                    0.2  & 0.8  & 0.6  & 6986  & 0     \\
                    0.2  & 0.8  & 0.7  & 6992  & 6     \\
                    0.2  & 0.8  & 0.8  & 6986  & 0     \\
                    0.2  & 0.8  & 0.9  & 6986  & 0     \\
                    0.2  & 0.8  & 1    & 6986  & 0     \\
                    0.3  & 0.7  & 0    & 6986  & 0     \\
                    0.3  & 0.7  & 0.1  & 6986  & 0     \\
                    0.3  & 0.7  & 0.2  & 7139  & 153   \\
                    0.3  & 0.7  & 0.3  & 7139  & 153   \\
                    0.3  & 0.7  & 0.4  & 6986  & 0     \\
                    0.3  & 0.7  & 0.5  & 6986  & 0     \\
                    0.3  & 0.7  & 0.6  & 6986  & 0     \\
                \end{tabular}}
                \label{tab:log2}
        \end{center}
    \end{minipage}
    \hfill
    \begin{minipage}[!h]{0.50\hsize}\centering
        \begin{center}\resizebox{0.8\textwidth}{!}{%
                %\caption{Лог работы программы.}
                \begin{tabular}{|c@{\hspace{5mm}}|c@{\hspace{5mm}}|c@{\hspace{5mm}}|c@{\hspace{5mm}}|c@{\hspace{5mm}}|c|}
                    \alpha        & \beta      & \rho      &Длина  & Разница \\
                    \hline
                    0.3  & 0.7  & 0.7  & 6986  & 0     \\
                    0.3  & 0.7  & 0.8  & 6992  & 6     \\
                    0.3  & 0.7  & 0.9  & 6992  & 6     \\
                    0.3  & 0.7  & 1    & 6986  & 0     \\
                    0.4  & 0.6  & 0    & 6986  & 0     \\
                    0.4  & 0.6  & 0.1  & 6992  & 6     \\
                    0.4  & 0.6  & 0.2  & 6986  & 0     \\
                    0.4  & 0.6  & 0.3  & 6986  & 0     \\
                    0.4  & 0.6  & 0.4  & 6986  & 0     \\
                    0.4  & 0.6  & 0.5  & 6992  & 6     \\
                    0.4  & 0.6  & 0.6  & 6992  & 6     \\
                    0.4  & 0.6  & 0.7  & 6986  & 0     \\
                    0.4  & 0.6  & 0.8  & 7139  & 153   \\
                    0.4  & 0.6  & 0.9  & 6986  & 0     \\
                    0.4  & 0.6  & 1    & 6992  & 6     \\
                    0.5  & 0.5  & 0    & 7139  & 153   \\
                    0.5  & 0.5  & 0.1  & 6986  & 0     \\
                    0.5  & 0.5  & 0.2  & 6986  & 0     \\
                    0.5  & 0.5  & 0.3  & 7139  & 153   \\
                    0.5  & 0.5  & 0.4  & 6986  & 0     \\
                    0.5  & 0.5  & 0.5  & 6986  & 0     \\
                    0.5  & 0.5  & 0.6  & 6986  & 0     \\
                    0.5  & 0.5  & 0.7  & 6986  & 0     \\
                    0.5  & 0.5  & 0.8  & 6986  & 0     \\
                    0.5  & 0.5  & 0.9  & 6986  & 0     \\
                    0.5  & 0.5  & 1    & 6986  & 0     \\
                    0.6  & 0.4  & 0    & 7139  & 153   \\
                    0.6  & 0.4  & 0.1  & 6992  & 6     \\
                    0.6  & 0.4  & 0.2  & 6986  & 0     \\
                    0.6  & 0.4  & 0.3  & 6986  & 0     \\
                    0.6  & 0.4  & 0.4  & 7139  & 153   \\
                    0.6  & 0.4  & 0.5  & 6992  & 6     \\
                    0.6  & 0.4  & 0.6  & 6986  & 0     \\
                    0.6  & 0.4  & 0.7  & 6986  & 0     \\
                    0.6  & 0.4  & 0.8  & 6986  & 0     \\
                    0.6  & 0.4  & 0.9  & 6992  & 6     \\
                    0.6  & 0.4  & 1    & 6986  & 0     \\
                    0.7  & 0.3  & 0    & 6986  & 0     \\
                    0.7  & 0.3  & 0.1  & 6986  & 0     \\
                    0.7  & 0.3  & 0.2  & 6986  & 0     \\
                    0.7  & 0.3  & 0.3  & 7139  & 153   \\
                \end{tabular}}
                %\label{T:log}
        \end{center}
    \end{minipage}
\end{table}
\clearpage
\begin{table}[!h]
    \begin{center}
        \begin{tabular}{|c@{\hspace{7mm}}|c@{\hspace{7mm}}|c@{\hspace{7mm}}|c@{\hspace{7mm}}|c@{\hspace{7mm}}|c|}
            \alpha        & \beta      & \rho      &Длина  & Разница \\
            \hline
            0.7  & 0.3  & 0.4  & 7165  & 179   \\
            0.7  & 0.3  & 0.5  & 7139  & 153   \\
            0.7  & 0.3  & 0.6  & 6992  & 6     \\
            0.7  & 0.3  & 0.7  & 6992  & 6     \\
            0.7  & 0.3  & 0.8  & 6986  & 0     \\
            0.7  & 0.3  & 0.9  & 6992  & 6     \\
            0.7  & 0.3  & 1    & 6986  & 0     \\
            0.8  & 0.2  & 0    & 7139  & 153   \\
            0.8  & 0.2  & 0.1  & 7562  & 576   \\
            0.8  & 0.2  & 0.2  & 6992  & 6     \\
            0.9  & 0.1  & 0.2  & 6992  & 6     \\
            0.9  & 0.1  & 0.3  & 6986  & 0     \\
            0.9  & 0.1  & 0.4  & 7139  & 153   \\
            0.9  & 0.1  & 0.5  & 7329  & 343   \\
            0.9  & 0.1  & 0.6  & 7217  & 231   \\
            0.9  & 0.1  & 0.7  & 7139  & 153   \\
            0.9  & 0.1  & 0.8  & 7217  & 231   \\
            0.9  & 0.1  & 0.9  & 7376  & 390   \\
            0.9  & 0.1  & 1    & 6986  & 0     \\
            1    & 0    & 0    & 8531  & 1545  \\
            1    & 0    & 0.1  & 8588  & 1602  \\
            1    & 0    & 0.2  & 6986  & 0     \\
            1    & 0    & 0.3  & 7720  & 734   \\
            1    & 0    & 0.4  & 7554  & 568   \\
            1    & 0    & 0.5  & 6992  & 6     \\
            1    & 0    & 0.6  & 7920  & 934   \\
            1    & 0    & 0.7  & 7217  & 231   \\
            1    & 0    & 0.8  & 7874  & 888   \\
            1    & 0    & 0.9  & 7446  & 460   \\
            1    & 0    & 1    & 8119  & 1133  \\
            \hline
        \end{tabular}
    \end{center}
\end{table}

\clearpage

\section*{Вывод}
В результате сравнения алгоритма полного перебора и муравьиного алгоритма по времени из таблицы \ref{tab:time} были получены следующие результаты:
\begin{itemize}
    \item при относительно небольших размерах матрицы смежности (а именно от 3 до 8) алгоритм полного перебора работает значительно быстрее (при размере 3 --- $\approx$ в 50 раз, при размере 6 --- $\approx$ в 60 раз);
    \item при размерах матрицы смежности 9 и выше, время работы алгоритма полного перебора начинает резко возрастать, и становится при размере 9 на 78\% медленнее муравьиного алгоритма, а на размере 10 алгоритм полного перебора работает $\approx$ в 12 раз дольше, нежели муравьиный алгоритм.
\end{itemize}

На основе проведенной параметризации для двух классов данных можно сделать следующие выводы:
\begin{itemize}
    \item Для класса данных №1, содержащего приблизительно равные значения, наилучшими наборами стали $(\alpha = 0.5, \beta = 0.5, \rho = \text{любое})$, так как они показали наиболее стабильные результаты, равные эталонному значению оптимального пути, равного 52 единицам.
    \item Для класса данных №2, содержащего различные значения, наилучшими наборами стали $(\alpha = 0.5, \beta = 0.5, \rho = \text{любое})$. При этих параметрах, количество найденных эталонных оптимальных путей составило 8 единиц.
\end{itemize}
